%24.10.2024, lecture 4

\subsection{Goldreich-Levin Theorem}
\goldreichlevin*
\begin{proof}[Proof of \Cref{thm:goldreich_levin}] \label{proof:goldreich_levin}
	It is easy to see that $\tilde f$ is also a strongly one-way function, since we didn't make it easier to invert.
    Assume we have an adversary that computes $B(x, r) = \langle x, r \rangle \mod 2 = \bigoplus_i x_i r_i$ from $f(x)$ and $r$. 
    Denote this adversary as $\tilde{B}(f(x), r)$.
    We will attempt to invert $f$.

    For some notation, consider strings as bit vectors.
    Let $e_i \coloneqq (0, 0, \ldots, 0, \underbrace{1}_i, 0, \ldots, 0)$, the vector with a single 1 in the $i$-th position and 0 elsewhere.

    Then, we can express $x_i$ as:
    \[
        x_i = \langle x, r \rangle \oplus \langle x, r \oplus e_i \rangle = B(x, r) \oplus B(x, r \oplus e_i).
    \] 
    However, we cannot simply use $\tilde{B}(f(x), r) \oplus \tilde{B}(f(x), r \oplus e_i)$, since these computations might not be independent.

    Define $\beta_r = \tilde{B}(f(x), r)$ and $\beta_{r \oplus e_i} = \tilde{B}(f(x), r \oplus e_i)$.
    We could attempt to guess $\beta_{r \oplus e_i}$ and use $\beta_r$, but we lack a way to verify its correctness without knowing the full $x$.
    Instead, we will guess values $\beta_s$ for a logarithmic number of vectors $s$ and then verify these guesses, leveraging the fact that $2^{O(\log n)} = n^{O(1)}$, so this approach remains computationally feasible.

    Sample $l = O(\log n)$ vectors $r^j$ uniformly from $U_n$.
    Try all possible bit values $\beta^j$ for $B(x, r^j)$, assuming that our guesses are correct.
    For each non-empty set $J \subseteq \{1, \ldots, l\}$, compute
    \[
        r^J = \bigoplus_{j \in J} r^j, \qquad \beta^J = \bigoplus_{j \in J} \beta^j.
    \] 

	Assuming that we've guessed all $\beta^{j}$ correctly it is clear that $\beta^{J} = B(x, r^{J})$, since
	\[
		B(x, r^{J}) = \left< x, r^{J} \right> = \bigoplus_{j \in J} \left< x, r^{j} \right> = \bigoplus_{j \in J} \beta^{j} = \beta^{J}
	.\] 
    Therefore, we can compute 
    \[
        x_{i}^J = \beta^J \oplus \tilde{B}(f(x), r^J \oplus e_i),
    \] with high probability.
    Let $\tilde{x}_i = \maj\limits_J x_i^J$ (the majority value across different sets $J$).
    We will now show that we have recovered the correct input with high probability.

    Suppose that $\tilde{B}$ succeeds with probability $\frac{1}{2} + \delta$, where $\delta = \frac{1}{n^{k}}$.

	\begin{lemma}
		Many inputs of $\tilde{B}$ are correct with high probability:
		 \[
			 \left|\left\{x \in \{0,1\}^n \mid \Pr[\tilde{B}(f(x), r) = B(x, r)] \geq \frac{1}{2} + \frac{\delta}{2}\right\}\right| \geq \delta \cdot 2^n.
		\] 
	\end{lemma}

	\begin{proof}
		  Define 
		  \[
			  S_n = \left\{x \in \{0,1\}^n \mid \Pr\left[\underbrace{\tilde{B}(f(x), r) = B(x, r)}_{S(x)}\right] \geq \frac{1}{2} + \frac{\delta}{2}\right\}.
		  \] 
		  We will use a counting argument:
		\[
			|U_n \setminus S_n| = 2^{n} \cdot \Pr_x \left[S(x) < \frac{1}{2} + \frac{\delta}{2}\right] = 2^{n} \cdot \Pr_x \left[1 - S(x) \geq \underbrace{\frac{1}{2} - \frac{\delta}{2}}_{\alpha^*}\right],
		\] 
		which implies that
		\[
			\E_x[1 - S(x)] = 1 - \left(\frac{1}{2} + \delta\right) = \frac{1}{2} - \delta.
		\] 
		Using Markov's inequality, we obtain:
		\[
			\Pr_x \left[1 - S(x) \geq \frac{1}{2} - \frac{\delta}{2}\right] \leq \frac{\frac{1}{2} - \delta}{\frac{1}{2} - \frac{\delta}{2}} = \frac{1 - 2 \delta}{1 - \delta} = 1 - \frac{\delta}{1 - \delta} \leq 1 - \delta.
		\] 
	\end{proof}

	\begin{lemma}
		$r^J \gets U_n$, and $r^J$ and $r^K$ are independent for $J \neq K$.
	\end{lemma}

	\begin{proof}
		Since each $r^{j} \gets  U_n$ it is clear that $r^{J} \gets  U_n$.
		If $K \subseteq J$, then
		 \begin{align*}
			 \Pr[r^J = t, r^K = t'] &= \Pr[r^{J \setminus K} = t \oplus t', r^{K} = t'] \\
									&= \Pr[r^{J \setminus K} = t \oplus t'] \cdot \Pr[r^K = t'] = \Pr[r^J = t] \cdot \Pr[r^K = t'],
		 \end{align*} 
		 where the last equality follows from the fact that $r^J$ are distributed uniformly. 

		 Otherwise, if $J \setminus K \neq \varnothing$ and $K \setminus J \neq \varnothing$, then
		 \begin{align*}
			 \Pr[r^J = t, r^K = t'] &= \sum_{t''} \Pr[r^J = t, r^K = t', r^{J \cap K} = t''] \\
									&= \sum_{t''} \Pr[r^{J \setminus K} = t \oplus t'', r^{K \setminus J} = t' \oplus t'', r^{J \cap K} = t''] \\
									&= \sum_{t''} \Pr[r^{J \setminus K} = t \oplus t''] \cdot \Pr[r^{K \setminus J} = t' \oplus t''] \cdot \Pr[r^{J \cap K} = t''] \\
									&= \Pr[r^{J} = t] \cdot \Pr[r^{K} = t'] \cdot \underbrace{\sum_{t''} \Pr[r^{J \cap K} = t'']}_{=1}.
		 \end{align*}
	\end{proof}

	Let $\xi_i^J \coloneqq [x_i = x_i^J]$ be the event that we decoded $x_i$ correctly.
	Now, since $\delta = \frac{1}{n^{k}}$, put $l = (2k + 2) \log_2 n$ and $m = 2^{l} - 1$.

	\begin{lemma}
	   For any $x \in S_n$, $i \in [n]$, and sufficiently large $n$, we have:
		\[
			\Pr\left[\sum_J \xi_i^J \leq \frac{m}{2}\right] < \frac{1}{2n}.
		\] 
	\end{lemma}

	\begin{proof}
		Since for any $x \in  S_n$ we have that probability of a success is at least $\frac{1}{2} + \frac{\delta}{2}$, then it is straightforward to see that
		\[
			\E\left[\sum_J \xi_i^J\right] \geq m\left(\frac{1}{2} + \frac{\delta}{2}\right).
		\]
		Given pairwise independence, we have $\Var(\sum_J \xi_i^J) = m \cdot \Var(\xi_i^J) \leq m$ (since the variance of Boolean random variable is no more than 1).
		Applying Chebyshev's inequality (in the form that $\Pr[\alpha < \E \alpha - \Delta] < \Var(\alpha) / \Delta^2$), we obtain:
		\[
			\Pr\left[\sum_J \xi_i^J < m\left(\frac{1}{2} + \frac{\delta}{2}\right) - \frac{m\delta}{2}\right] < \frac{4m \Var(\xi_i^J)}{m^2 \delta^2} \leq \frac{4}{m \delta^2} \leq \frac{4}{n^2} \leq \frac{1}{2n}.
		\] 
	\end{proof}

	With these lemmas in place, we try all possible values of $\beta^j$ for the base vectors $r^j$.
	For the correct choices of $\beta$, most values $x_i^J$ are correct with high probability.
	Therefore, we compute the correct $x_i$ using the majority vote with probability $\geq \frac{1}{2}$, achieving high probability overall.
	Once we have determined all $x_i$ values, we verify that we have correctly decoded $f(x)$.
\end{proof}

All of the exercises below relate to the proof above.

\begin{exercise}
    Is it necessary for the one-way function (OWF) to be length-preserving?
\end{exercise}

\begin{exercise}
    In the proof, we rely on knowing the success probability of $\tilde{B}$.
    Can we reformulate the proof to avoid this requirement?
\end{exercise}

\begin{exercise}
    How would the proof change if $f$ is not injective?
\end{exercise}

\section{Pseudorandom Generators}

\begin{definition}
  We say that $G \colon \{0,1\}^l \to \{0,1\}^{f(l)}$ is an $f(l)$-PRG if, for every adversary $A$, the distributions $G(U_l)$ and $U_{f(l)}$ are computationally indistinguishable.
\end{definition}

If one creates such a generator for $f(l) = 2^{l}$, this implies that all polynomial-time randomized algorithms are contained in $\PP$!

Recall our public-key encryption scheme:
\[
E^{**}(b_1, \ldots, b_m, e, r) = \left(e^{m}(r), B(r) \oplus b_1, B(e(r)) \oplus b_2, \ldots, B(e^{m - 1}(r)) \oplus b_m\right).
\] 

\begin{theorem}
  If one can break $E^{**}$ (i.e., distinguish $0\ldots0$ from a random message), then one can also break a PRG.
\end{theorem}

Here, $e$ is a public function, while $d$ is a private inverse function of $e$.
The function $B$ is a hardcore predicate.
Alice can decrypt the message using $d^{m}(e^{m}(r)) = r$, allowing her to compute the other bits.

\begin{lemma}
	If $g$ is a length-preserving one-way permutation (OWP) and $B$ is its hardcore predicate, then there exists an $f(l)$-PRG, defined by
	 \[
	  G_{f(l)}(x) = g^{f(l) - l}(x) \circ B(x) \circ B(g(x)) \circ \dots \circ B(g^{f(l) - l - 1}(x)).
	 \] 
\end{lemma}
This construction extends the input by $i$ bits.
In the proof below, $U$ denotes a uniform distribution of strings of fixed length, though $U$ may refer to different lengths depending on the context.

\begin{proof}
  Idea: if we can distinguish $U$ from $G_{f(l)}(U)$, then there exists an index $i$ where we can distinguish between $G_i(U)$ and $G_{i + 1}(U)$.

  Consider the case where one can distinguish between $U$ and $G_{1}(U)$.
  Suppose that $G(x) = G_1(x) = g(x) \circ B(x)$ is broken by some adversary $A$:
   \[
	   \left| \underbrace{\Pr[A(G(x)) = 1]}_{\gamma} - \underbrace{\Pr[A(y) = 1]}_{\nu} \right| \ge \frac{1}{l^{k}}
  ,\] 
  where $y$ is distributed uniformly with length $|y| = |x| + 1$.
  
  Define two variables:
   \begin{align*}
	   \alpha &\coloneqq \Pr[A(g(x) \circ b) = 1  \mid b = B(x)] = \Pr[A(g(x) \circ B(x)) = 1] = \gamma,\\
	   \beta &\coloneqq \Pr[A(g(x) \circ b) = 1  \mid b \neq  B(x)] = \Pr[A(g(x) \circ \overline{B(x)}) = 1].
   \end{align*}
  Thus,
  \[
	  \nu = \Pr[A(y) = 1] = \Pr[A(g(x) \circ b) = 1] = \Pr[b = B(x)] \cdot \alpha + \Pr[b \neq B(x)] \cdot \beta = \frac{\alpha + \beta}{2}.
  \] 
  Without loss of generality, we assume that $\frac{1}{l^{k}} \le \gamma - \nu = \alpha - \frac{\alpha + \beta}{2} = \frac{\alpha - \beta}{2}$ (otherwise, we simply flip all outputs of $A$).
  
  Our new adversary $A'$ computes $B(x)$ from $f(x)$: it picks a random bit $b \gets U_1$ and, if $A(f(x) \circ b) = 1$, outputs $b$, otherwise it outputs $\overline{b}$.
  
  The probability of success is:
  \begin{align*}
	  \Pr[A'(f(x)) = B(x)] &= \Pr[b = B(x)] \cdot \Pr[A(f(x) \circ B(x)) = 1] + \\
						   &+ \Pr[b = \overline{B(x)}] \cdot \Pr[A(f(x) \circ \overline{B(x)}) \neq 1] \\
						   &= \frac{1}{2} \alpha + \frac{1}{2} (1 - \beta) = \frac{1}{2} + \frac{\alpha - \beta}{2} \ge  \frac{1}{2} + \frac{1}{l^{k}}.
  \end{align*}
  Thus, we can break $B$.

  Now, suppose that an adversary $A$ can distinguish between $G_i(U)$ and $G_{i + 1}(U)$. \todo{it is not correct, fix}
  We will show that this implies an ability to distinguish between $U$ and $G_1(U)$.
  
  Let $x$ and $z$ be random variables (uniformly distributed) such that $|z| = |x| + 1$.
  Then $A$ can distinguish between:
   \begin{align*}
	   G_{i + 1}(x) &= g^{i + 1}(x) \circ B(x) \circ B(g(x)) \circ \dots \circ B(g^{i}(x)),\\
	   G_{i}(z) &= g^{i}(z) \circ B(z) \circ B(g(z)) \circ \dots \circ B(g^{i - 1}(z)).
   \end{align*}
  Hence, $A$ distinguishes between $g^{i + 1}(x) \circ B(x)$ and $g^{i}(z) \gets U$.
\end{proof}
